
  
\documentclass[11pt,a4paper]{article}
\author{OFFLUCK}

\usepackage[utf8]{inputenc}
\usepackage[russian]{babel}
\usepackage[OT1]{fontenc}
\usepackage{amsmath}
\usepackage{amsfonts}
\usepackage{amssymb}
\usepackage[left=2cm,right=2cm,top=2cm,bottom=2cm]{geometry}
\usepackage[colorlinks=true,urlcolor=blue]{hyperref}
\usepackage{xcolor}




\setlength{\parskip}{1em}
\setlength{\parindent}{0pt}

\usepackage{fancyhdr}
\pagestyle{fancy}
\fancyhf{}
\fancyhead[L]{\textsf{\textbf{Зачёт.} Мини-вопросы.}}
\fancyhead[R]{\textsf{\href{https://t.me/olegsama}{Сидоренков Олег}}}
\fancyfoot[C]{\textsf{Высшая школа экономики, ОП <<Программная инженерия>>}}

\renewcommand{\headrulewidth}{0.3pt}
\renewcommand{\footrulewidth}{0.3pt}

\begin{document}

\begin{center}
    \begin{huge}
        \textsf{Телекоммуникационные технологии. Мини-вопросы.}
    \end{huge}
\end{center}

\textbf{1). Что такое инкапсуляция.}
\\
Это процесс передачи данных с верхнего уровня приложений вниз (по стеку протоколов) к физическому уровню, чтобы быть переданными по сетевой физической среде
\\
\textbf{2). Для кого предназначены пакеты, имеющие адрес назначения 255.255.255.255.}
\\
Всем устройствам в локальной сети.
\\
\textbf{3). Зачем нужна фрагментация пакетов.}
\\
Длина пакета может достигать 64КБ, что может превышать размер фрейма протокола нижнего уровня, в который инкапсулируется пакет. Поскольку пакет может передаваться по средам с разными значениями фрейма, в него был встроен механизм фрагментации.
\\
\textbf{4). Как отправитель узнает, по какому маршруту отправлять данные в протоколе AODV.}
\\
Когда один из узлов пытается отправить данные в сеть, посылается пакет с запросом на установку маршрута RREQ. Другие узлы сети с AODV пересылают этот пакет в общую среду и делают запись об узле, от которого они приняли запрос, создавая массовую отправку временных маршрутов к запрашивающему узлу. Когда узел получает запрос RREQ и уже имеет маршрут к узлу назначения, то в зависимости от флага «D» сообщения RREQ, он посылает назад сообщение RREP через временный маршрут к узлу требования (узлу-инициатору маршрута) или направляет сообщение RREQ к узлу-получателю, который отправляет сообщение RREP назад узлу-отправителю. Запрашивающий узел в таком случае использует маршрут с наименьшим количеством промежуточных узлов.
\\
Флаг «D» — устанавливается, если необходимо доставить сообщение RREQ непосредственно до узла-получателя.
\\
\textbf{5). Какая основная причина потери пакетов в проводных сетях. 
}
\\
Перегрузка сети, то есть, когда достигается максимальная пропускная способность.
\\
Поскольку процесс пакетной передачи следует определенным шагам, сбои в соединениях могут привести к потере некоторых пакетов.
\\
\textbf{6). Какая идея лежит в алгоритмах сжатия без потерь.}
\\
Основной принцип алгоритмов сжатия базируется на том, что в любом файле, содержащем неслучайные данные, информация частично повторяется.
\\
Используя статистические математические модели можно определить вероятность повторения определённой комбинации символов.
\\
После этого можно создать коды, обозначающие выбранные фразы, и назначить самым часто повторяющимся фразам самые короткие коды.
\\
\textbf{7). Что такое маскировка звука.}
\\
Это явление, при котором порог слышимости определённого звука повышается под влиянием других звуков.
\\
\textbf{8). Что такое аналоговый сигнал. }
\\
Аналоговый сигнал - это непрерывный электрический сигнал.
\\
\textbf{9). Что такое цифровой сигнал.}
\\
Цифровой сигнал - это сигнал, который можно представить в виде последовательности дискретных (цифровых) значений.
\\
Получается он посредством приведения аналогово сигнала к дискретному, который может принимать какие-то определённые значения.  
\\
\textbf{10). Какой протокол транспортного уровня использует сокет без установления соединения.}
\\
UDP (User Datagram Protocol)
\\
\textbf{11). В чем разница между статической и динамической веб-страницей.}
\\
Статическая веб-страница - это чаще всего несколько html-файлов. Пользователь не может повлиять на поведение статической страницы.
\\
Динамическая же состоит в основном из html и js файлов. Она позволяет пользователю взаимодейтсвовать с контентом и изменять содержимое страницы.
\\
\textbf{12). Какой версии http соответствует обозначение *https://* в веб странице.}
\\
HTTP1.1. HTTPS - это протокол, состоящий из двух: HTTP и SSL/TSL.
\\
HTTPS - это расширение протокола HTTP для поддержки шифрования в целях повышения безопасности. Данные в протоколе HTTPS передаются поверх криптографических протоколов TLS или устаревшего в 2015 году SSL
\\
\textbf{13). Недостатки симметричной криптографии.}
\\
В отличие от ассиметричного шифрования, здесь один ключ используется как для шифрования сообщения, так и для его дешифрования.
\\
Таким образом, если ключ будет перехвачен в незащищённом канале, злоумышленники с лёгкостью могут им воспользоваться.
\\
Остаётся только передавать ключ максимально безопасным способом, что часто не так легко реализовать.
\\
\textbf{14). Что такое мгновенная отправка данных в QUIC.}
\\
В случае потеря пакетов по HTTP / 2 через TCP , это может вызвать серьезные проблемы с производительностью. Когда посылка утеряна, получатель должен дождаться ее восстановления.
\\
Для решения этой проблемы протокол QUIC позволяет потокам данных независимо достигать получателя. Нет необходимости ждать, пока эти потерянные пакеты данных будут извлечены, и, следовательно, это не вызывает особого беспокойства, если это произойдет.
\\
\textbf{15). Разница между ЭЦП и шифрованием на открытых ключах.}
\\
Разница состоит в том, что шифрование используется для того, чтобы передать зашифрованное сообщение, которое не должно попасть в руки злоумышленнику в изначальном виде.
\\
А ЭПЦ позволяет подтвердить надёжность источника, из которого было получено сообщение.
\\
\textbf{16). Теорема Котельникова и почему частота дискретизации звуковых файлов 44.1 кГц.}
По теореме Котельникова, если сигнал имеет ограниченный спектр, то есть все его частоты не больше какого-то числа $F_{max}$, любая функция S(t), являющаяся аналоговым сигналом, определяется последовательностью мнгновенных значений $\Delta t \leq \frac{1}{2 \cdot F_{max}}$.
\\
\end{document}
